%% For double-blind review submission, w/o CCS and ACM Reference (max submission space)
\documentclass[acmsmall,review,anonymous]{acmart}\settopmatter{printfolios=true,printccs=false,printacmref=false}
%% For double-blind review submission, w/ CCS and ACM Reference
%\documentclass[acmsmall,review,anonymous]{acmart}\settopmatter{printfolios=true}
%% For single-blind review submission, w/o CCS and ACM Reference (max submission space)
%\documentclass[acmsmall,review]{acmart}\settopmatter{printfolios=true,printccs=false,printacmref=false}
%% For single-blind review submission, w/ CCS and ACM Reference
%\documentclass[acmsmall,review]{acmart}\settopmatter{printfolios=true}
%% For final camera-ready submission, w/ required CCS and ACM Reference
%\documentclass[acmsmall]{acmart}\settopmatter{}


%% Journal information
%% Supplied to authors by publisher for camera-ready submission;
%% use defaults for review submission.
\acmJournal{PACMPL}
\acmVolume{1}
\acmNumber{CONF} % CONF = POPL or ICFP or OOPSLA
\acmArticle{1}
\acmYear{2018}
\acmMonth{1}
\acmDOI{} % \acmDOI{10.1145/nnnnnnn.nnnnnnn}
\startPage{1}

%% Copyright information
%% Supplied to authors (based on authors' rights management selection;
%% see authors.acm.org) by publisher for camera-ready submission;
%% use 'none' for review submission.
\setcopyright{none}
%\setcopyright{acmcopyright}
%\setcopyright{acmlicensed}
%\setcopyright{rightsretained}
%\copyrightyear{2018}           %% If different from \acmYear

%% Bibliography style
\bibliographystyle{ACM-Reference-Format}
%% Citation style
%% Note: author/year citations are required for papers published as an
%% issue of PACMPL.
\citestyle{acmauthoryear}   %% For author/year citations


%%%%%%%%%%%%%%%%%%%%%%%%%%%%%%%%%%%%%%%%%%%%%%%%%%%%%%%%%%%%%%%%%%%%%%
%% Note: Authors migrating a paper from PACMPL format to traditional
%% SIGPLAN proceedings format must update the '\documentclass' and
%% topmatter commands above; see 'acmart-sigplanproc-template.tex'.
%%%%%%%%%%%%%%%%%%%%%%%%%%%%%%%%%%%%%%%%%%%%%%%%%%%%%%%%%%%%%%%%%%%%%%


%% Some recommended packages.
\usepackage{booktabs}   %% For formal tables:
                        %% http://ctan.org/pkg/booktabs
\usepackage{subcaption} %% For complex figures with subfigures/subcaptions
                        %% http://ctan.org/pkg/subcaption


\begin{document}

%% Title information
\title[Short Title]{Full Title}         %% [Short Title] is optional;
                                        %% when present, will be used in
                                        %% header instead of Full Title.
\titlenote{with title note}             %% \titlenote is optional;
                                        %% can be repeated if necessary;
                                        %% contents suppressed with 'anonymous'
\subtitle{Subtitle}                     %% \subtitle is optional
\subtitlenote{with subtitle note}       %% \subtitlenote is optional;
                                        %% can be repeated if necessary;
                                        %% contents suppressed with 'anonymous'


%% Author information
%% Contents and number of authors suppressed with 'anonymous'.
%% Each author should be introduced by \author, followed by
%% \authornote (optional), \orcid (optional), \affiliation, and
%% \email.
%% An author may have multiple affiliations and/or emails; repeat the
%% appropriate command.
%% Many elements are not rendered, but should be provided for metadata
%% extraction tools.

%% Author with single affiliation.
\author{First1 Last1}
\authornote{with author1 note}          %% \authornote is optional;
                                        %% can be repeated if necessary
\orcid{nnnn-nnnn-nnnn-nnnn}             %% \orcid is optional
\affiliation{
  \position{Position1}
  \department{Department1}              %% \department is recommended
  \institution{Institution1}            %% \institution is required
  \streetaddress{Street1 Address1}
  \city{City1}
  \state{State1}
  \postcode{Post-Code1}
  \country{Country1}                    %% \country is recommended
}
\email{first1.last1@inst1.edu}          %% \email is recommended

%% Author with two affiliations and emails.
\author{First2 Last2}
\authornote{with author2 note}          %% \authornote is optional;
                                        %% can be repeated if necessary
\orcid{nnnn-nnnn-nnnn-nnnn}             %% \orcid is optional
\affiliation{
  \position{Position2a}
  \department{Department2a}             %% \department is recommended
  \institution{Institution2a}           %% \institution is required
  \streetaddress{Street2a Address2a}
  \city{City2a}
  \state{State2a}
  \postcode{Post-Code2a}
  \country{Country2a}                   %% \country is recommended
}
\email{first2.last2@inst2a.com}         %% \email is recommended
\affiliation{
  \position{Position2b}
  \department{Department2b}             %% \department is recommended
  \institution{Institution2b}           %% \institution is required
  \streetaddress{Street3b Address2b}
  \city{City2b}
  \state{State2b}
  \postcode{Post-Code2b}
  \country{Country2b}                   %% \country is recommended
}
\email{first2.last2@inst2b.org}         %% \email is recommended


%% Abstract
%% Note: \begin{abstract}...\end{abstract} environment must come
%% before \maketitle command
\begin{abstract}
Transformations and musical patterns -> Haskell -> 
\end{abstract}


%% 2012 ACM Computing Classification System (CSS) concepts
%% Generate at 'http://dl.acm.org/ccs/ccs.cfm'.
\begin{CCSXML}
<ccs2012>
<concept>
<concept_id>10011007.10011006.10011008</concept_id>
<concept_desc>Software and its engineering~General programming languages</concept_desc>
<concept_significance>500</concept_significance>
</concept>
<concept>
<concept_id>10003456.10003457.10003521.10003525</concept_id>
<concept_desc>Social and professional topics~History of programming languages</concept_desc>
<concept_significance>300</concept_significance>
</concept>
</ccs2012>
\end{CCSXML}

\ccsdesc[500]{Software and its engineering~General programming languages}
\ccsdesc[300]{Social and professional topics~History of programming languages}
%% End of generated code


%% Keywords
%% comma separated list
\keywords{transformation, edit distance, musical patterns, evaluation,
  clustering, ...}  %% \keywords are mandatory in final camera-ready submission


%% \maketitle
%% Note: \maketitle command must come after title commands, author
%% commands, abstract environment, Computing Classification System
%% environment and commands, and keywords command.
\maketitle


\section{Introduction}
\paragraph{Musical patterns are hard to define}
Musical patterns are generated by composers \cite{}, employed by performers
\cite{}, perceived by listeners \cite{}, and help us better understand and
communicate about music. However, in music theory and music
information retrieval (MIR), the definition of ``musical pattern'' has been
elusive. For example, one can argue a musical pattern can be ``an excerpt
of special importance'', ``a salient fragment'', ``a prominent unit'', etc.. In
addition, in different corpora, experts use terminologies such as ``lick'',
``riff'', ``sequence'', etc.. The variability in the definition poses
difficulties on designing and evaluating automated computational systems to extract musical patterns.

\paragraph{Musical patterns are useful but hard to extract} Patterns in music are relevant in many
musical activities: created by composers\ \cite{}, used by performers \cite{}
and perceived by listeners \cite{}, people conceptualise \cite{}, comprehend
\cite{} and communicate meaningful patterns in the context of music. One could
try a data-driven way and define musical patterns as the data generated by such
activities. Nevertheless, another challenge emerge: the subjectivity and
ambiguity give rise to a demand is personalised data, therefore induce the curse
of high dimensionality. This also contribute to the vicious loop going from the
complexity of the task to a lack of annotations datasets in the field and back.
Therefore, with a wide potential for various applications, the automation of
musical pattern analysis and discovery is an important and difficult task. The
challenge is: How do we leverage the limited data and theory to design
and evaluate a pattern discovery system?

\paragraph{MIREX}
Previous research has addressed the challenge to a certain extent. Historically, algorithms have been tested on unassociated datasets with disparate metrics \cite{janssen2013finding}. One attempt to standardise the evaluation of algorithms is the \textsc{mirex} Discovery of Repeated Themes \& Sections task initiated in 2014. In the task, a pattern is defined as a set of time-pitch pairs that occurs at least twice in a piece of music and the \textsc{jku-pdd} dataset was introduced \cite{collins2013discovery}. According to the evaluation metrics in this task, the state-of-the-art algorithms perform acceptably well in precision, recall, and F1-scores, although they cannot reproduce the human-annotated patterns yet. Another pattern annotation dataset which has been used for evaluating the algorithms is the \textsc{mtc-ann} Dutch Folk Song dataset \cite{van2016meertens}: human-annotations have been compared with algorithmically extracted patterns by their performance in a classification task \cite{boot2016evaluating} showing the annotated patterns perform better. Furthermore, a large disagreement between annotated and computationally extracted patterns has been shown in both the \textsc{jku-pdd} and \textsc{mtc-ann} dataset in \cite{ren2017search,ren2018}.


\paragraph{Existing algorithms}
The algorithms submitted to MIREX use different models and methods from geometry
\cite{},information theory \cite{} and machine learning \cite{}. Often, the
algorithms provide pattern candidates instead of a definitive and
simple answer on why the algorithms identifies the output patterns.

\paragraph{Our Methods} In this paper, we encode well-defined, music and
computationally relevant transformations to investigate the output of pattern
discovery algorithms, human annotations and random passages. The implementation
in the functional programming language Haskell gives a clean representation of
the pattern comparison and extraction process.

\paragraph{Contributions}
- Implementation in Haskell
- Analysis on MTC-ANN and MIREX
- Detection and query

\section{Musical Patterns and Modelling}

\paragraph{Repetitions and Variations}
Musical patterns and variations are closely related.
Variation, in the generic sense of a change or slight difference, can be local
or global in music: ornaments such as trills and turns are local, form and
thematic variations are global. 

"variation is a formal technique where material is repeated in an altered form.
The changes may involve melody, rhythm, harmony, counterpoint, timbre,
orchestration or any combination of these."
-- wikipedia

\paragraph{Prototype and transformations}
Irrelevant to which type of variation it is, there must be an original material
whence the variations are developed and can be related back to.
We refer the original materials as the prototype patterns; we name the processes
of going from the original materials to the variations as a morphing processes
using operators to perform transformations.

Different types of transformations can be coded as functions $f: A -> B$, where
$A \in {Prototypes}$, $B \in {Variations}$.


For example, one simple transformation is
chromatic (real) transposition, where $f(x) = x + n, n \in {allowed pitch shifts}$.
Evidence in cognitive science confirms that this is a natural equivlance
relationship except for people with absolute/perfect pitch.

\section{Domain Specific Language in Haskell}

\paragraph{Haskell}
Combinator DSL.

Functors.

Contravariant functors.

Computationally combine viewpoints and transformations. With modified input,
stream computation in a mathematically found way. 

A spectrum of modifications using diff and edit distance.

\paragraph{Pattern discovery and querying system}

\subsection{Transformations}
- Edit distance: modelling using diff, fuzziness modelled with one parameter

\section{Music Materials and Patterns in Music}
We use the \textsc{mtc-ann} Dutch Folk Song dataset \cite{van2016meertens}, which contains an exceptionally large number of annotated patterns and is therefore suitable for a classification experiment. In this section, we examine groups of patterns, random passages, and their features in this dataset.

\vspace{-1mm}
\subsection{Pattern groups in \textsc{mtc-ann}}\label{sec:es}
\textbf{Annotated patterns}
During the making of \textsc{mtc-ann}, three experts have been asked to annotate the prominent patterns in each song which best classify the song into one of 26 tune families. \textit{Tune family} is a concept in ethnomusicology that groups together tunes sharing the same ancestor in the process of oral transmission \cite{10.2307/851236}. The dataset consists of 360 Dutch folk songs with 1657 annotated pattern occurrences. In an annotation study on what influences human judgements when categorising melodies belonging to the same tune family, repeated patterns turned out to play the most important role \cite{volk2012melodic}. It is, therefore, reasonable to use repeated pattern discovery algorithms on this dataset.

\textbf{Patterns from algorithms}
We use the six pattern discovery algorithms and extract the patterns from the \textsc{mtc-ann} dataset using the same setup as in \cite{ren2017search,boot2016evaluating}. The extracted patterns from each algorithm form a subgroup under the umbrella of the extracted pattern group. The seven algorithms were submitted to the \textsc{mirex} task during 2014-2017: \textsc{siatecc}ompress - \textsc{tlp} (\textbf{\textsc{siap}}), \textsc{siatecc}ompress - \textsc{tlf}\oldstylenums{1} (\textbf{\textsc{siaf1}}), \textsc{siatecc}ompress - \textsc{tlr} (\textbf{\textsc{siar}}) \cite{meredith2016using}, \textbf{\textsc{vm}} \& \textbf{\textsc{vm2}} \cite{velarde2014wavelet}, \textsc{symchm} (\textbf{\textsc{sc}}) \cite{pesek2015symchm}, and \textsc{siarct-cfp} (\textbf{\textsc{siacfp}}) \cite{collins2013siarct}.

We compare annotated and extracted patterns with randomly sampled passages as a baseline in order to potentially support or refuse the significance of musical patterns. In more detail, taking the annotated patterns from \textsc{mtc-ann}, random passages are sampled with the following procedures: for each annotated pattern, we find the corresponding song where the annotation appears. We then find a random starting point and take an excerpt of the same length as the pattern to construct a candidate excerpt. Finally, we repeat the sampling procedures five times to prevent accidental results.


\section{Results}

\subsection{Patterns Explained by Transformations}

\subsection{Comparing Algorithms}
Different profiles in different datasets.

\subsection{Pattern querying and discovery}

\section{Discussion}

Future work: Polyphonic


%% Acknowledgments
\begin{acks}                            %% acks environment is optional
                                        %% contents suppressed with 'anonymous'
  %% Commands \grantsponsor{<sponsorID>}{<name>}{<url>} and
  %% \grantnum[<url>]{<sponsorID>}{<number>} should be used to
  %% acknowledge financial support and will be used by metadata
  %% extraction tools.
  This material is based upon work supported by the
  \grantsponsor{GS100000001}{National Science
    Foundation}{http://dx.doi.org/10.13039/100000001} under Grant
  No.~\grantnum{GS100000001}{nnnnnnn} and Grant
  No.~\grantnum{GS100000001}{mmmmmmm}.  Any opinions, findings, and
  conclusions or recommendations expressed in this material are those
  of the author and do not necessarily reflect the views of the
  National Science Foundation.
\end{acks}


%% Bibliography
%\bibliography{bibfile}


%% Appendix
\appendix
\section{Appendix}

Text of appendix \ldots

\end{document}
